
Ethereum is a public blockchain based cryptocurrency and supports Turing-complete smart contracts~\cite{ethereum}.
The blockchain maintains a list of records, called \emph{blocks} and serves as a ledger for all transactions.
The value token of the blockchain is called \emph{ether} and 
is traded in cryptocurrency exchanges.
Clients use ether to pay for executing and maintaining the transactions.~\footnote{
\emph{Finney}, \emph{Szabo}, and \emph{Wei} are the denominations of ether.}
 
Ethereum blockchain is growing at a rate 1.2GB/month~\cite{ethdiskspace}.
As of June 2016, the space required by blockchain is as high as 61GB.
Compressing the blockchain can therefore be useful for full nodes that downloading the entire blockchain.

In this project, we propose to compress the entire ethereum blockchain. 
Although compressing blocks also involves EVM bytecode compression, 
we view them as complementary to one another.
Section~\ref{sec:blockcompress} and 
Section~\ref{sec:evmcompress}  
outline the methodology for compressing the blocks and EVM byte code.

\subsection{Blockchain Compression}\label{sec:blockcompress}

A block is comprised of block header and a series of transactions. A transaction is a signed message.
There are two types of transactions: those resulting in message calls and those which result in creating new contracts.
Both types have the following common fields:
\begin{description}
  \item[nonce:] A scalar value representing number of transactions associated with this account
  \item[gasPrice:] Default value is 0.02e12 Wei. Though clients can set their own gasPrice, most of the transactions use default value suggesting the high probability of its occurance.
  \item[gasLimit:] Maximum gas that can be used for executing this transaction
  \item[to:] Receiver's 160-bit address
  \item[value:] Wei being transferred to the receipient.
  \item[v,r,s:] v =\{27,28\} and so can be efficiently encoded.
\end{description}

Additionally, contract creation has \textbf{init} field containing EVM-bytecode for initializing the contract.
We propose to compress the EVM bytecode using techniques presented in Section~\ref{sec:evmcompress}. 

Message call transaction has \textbf{data} field containing the inputs of the message call.

%
\subsection{EVM ByteCode Compression}\label{sec:evmcompress}

There are several ways to compress EVM bytecode. Huffman encoding is probably the most straightforward
way to reduce bytecode size. By using Huffman encoding, we essentially compress instructions individually
by minimizing the number of bits taken by each opcode given how frequently they appear. One potential
benefit of using Huffman codes is that we can modify the EVM to directly execute the compressed bytecode,
without decoding the entire bytecode stream first. It, however, raises the challenge of making program counter modifications work
correctly, because now instructions are variable-length encoded.

Huffman codes reduce the size of individual instruction encoding, but do not take advantage of repetitive patterns in the bytecode.
A dictionary-based or grammar-based compression scheme can further reduce redundant information in the bytecode.
General-purpose compression algorithms like the LZ-family or Sequitur might just work,
but likely not optimal, because of the specific structure in bytecode sequences.
For example, some instructions take operands, sometimes very long operands.
Long operands are usually just random bits and are much harder to compress than the opcodes, which typically appear in patterns.
These long operands may confuse general-purpose compress algorithms, resulting in sub-optimal results.
We aim to devise a domain-specific compression scheme for EVM bytecode that can achieve better compression rates
by treating operands the opcodes separately.


