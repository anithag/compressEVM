
Ethereum is a public blockchain based cryptocurrency and supports Turing-complete smart contracts~\cite{ethereum}.
The blockchain maintains a list of records, called \emph{blocks} and serves as a ledger for all transactions.
A block is comprised of block headers and a series of transactions. A transaction is a signed message.
There are two types of transactions: those resulting in message calls and those which result
 in creating new contracts.
Both types have the following common fields:
\begin{description}
  \item[nonce:]
  \item[gasPrice:]
  \item[gasLimit:]
  \item[to:]
  \item[value:]
  \item[v,r,s:]
\end{description}

Contract creation has \textbf{init} field containing EVM-bytecode for initializing the contract.
Message call transaction has \textbf{data} field containing the inputs of the message call.

Ethereum blockchain is growing at a rate 1.2GB/month~\cite{ethdiskspace}.
As of June 2016, the space required by blockchain is as high as 61GB.
Compressing the blockchain can therefore be useful for full nodes that downloading the entire blockchain.

Although compressing blocks also involves EVM bytecode compression, we view them as logically separate tasks.
Section~\ref{sec:blockcompress} and 
Section~\ref{sec:evmcompress}  
outline the methodology for compressing the blocks and EVM byte code.

\subsection{Blockchain Compression}\label{sec:blockcompress}

%
\subsection{EVM ByteCode Compression}\label{sec:evmcompress}

There are several ways to compress EVM bytecode. Huffman encoding is probably the most straightforward
way to reduce bytecode size. By using Huffman encoding, we essentially compress instructions individually
by minimizing the number of bits taken by each opcode given how frequently they appear. One potential
benefit of using Huffman codes is that we can modify the EVM to directly execute the compressed bytecode,
without decoding the entire bytecode stream first. It, however, raises the challenge of making program counter modifications work
correctly, because now instructions are variable-length encoded.

Huffman codes reduce the size of individual instruction encoding, but do not take advantage of repetitive patterns in the bytecode.
A dictionary-based or grammar-based compression scheme can further reduce redundant information in the bytecode.
General-purpose compression algorithms like the LZ-family or Sequitur might just work,
but likely not optimal, because of the specific structure in bytecode sequences.
For example, some instructions take operands, sometimes very long operands.
Long operands are usually just random bits and are much harder to compress than the opcodes, which typically appear in patterns.
These long operands may confuse general-purpose compress algorithms, resulting in sub-optimal results.
We aim to devise a domain-specific compression scheme for EVM bytecode that can achieve better compression rates
by treating operands the opcodes separately.


