
Ethereum is a public blockchain based cryptocurrency and supports Turing-complete smart contracts~\cite{ethereum}.
The blockchain maintains a list of records, called \emph{blocks} and serves as a ledger for all transactions.
A block is comprised of block headers and a series of transactions. A transaction is a signed message.
There are two types of transactions: those resulting in message calls and those which result
 in creating new contracts.
Both types have the following common fields:
\begin{description}
  \item[nonce:]
  \item[gasPrice:]
  \item[gasLimit:]
  \item[to:]
  \item[value:]
  \item[v,r,s:]
\end{description}

Contract creation has \textbf{init} field containing EVM-bytecode for initializing the contract.
Message call transaction has \textbf{data} field containing the inputs of the message call.

Ethereum blockchain is growing at a rate 1.2GB/month.
%\cite{ethdiskspace}.
As of June 2016, the space required by blockchain is as high as 61GB.
Compressing the blockchain can therefore be useful for full nodes that downloading the entire blockchain.

Although compressing blocks also involves EVM bytecode compression, we view them as logically separate tasks.
Section~\ref{sec:blockcompress} and 
Section~\ref{sec:evmcompress}  
outline the methodology for compressing the blocks and EVM byte code.

\subsection{Blockchain Compression}\label{sec:blockcompress}


\subsection{EVM ByteCode Compression}\label{sec:evmcompress}

Assigning optimal byte codes can reduce the storage for byte code.
%\cite{aslam2010}. 
The EVM can be modified to directly execute the compressed byte code.
%\cite{latendresse2003}.
Write any novel contributions here.
