
Ethereum is a public blockchain based cryptocurrency and supports Turing-complete smart contracts~\cite{ethereum}.
The blockchain maintains a list of records, called \emph{blocks} and serves as a ledger for all transactions.
The value token of the blockchain is called \emph{ether} and 
is traded in cryptocurrency exchanges.
Clients use ether to pay for executing and maintaining the transactions.~\footnote{
\emph{Finney}, \emph{Szabo}, and \emph{Wei} are the denominations of ether.}
 
Ethereum blockchain is growing at a rate 1.2GB/month~\cite{ethdiskspace}.
As of June 2016, the space required by blockchain is as high as 61GB.
Compressing the blockchain can therefore be useful for full nodes that download the entire blockchain.

In this project, we propose to compress the entire ethereum blockchain. 
Although compressing blocks (and hence transactions) also involves EVM bytecode compression, 
we view them as complementary tasks. 
In particular techniques to compress bytecode are orthogonal to techniques for compressing other fields of transaction.
Section~\ref{sec:blockcompress} and 
Section~\ref{sec:evmcompress}  
outline the methodology for compressing the blocks and EVM bytecode.

\subsection{Blockchain Compression}\label{sec:blockcompress}

A block is comprised of block header and a series of transactions. A transaction is a signed message.
There are two types of transactions: those resulting in message calls and those which result in creating new contracts.
Both types have the following common fields:
\begin{description}
  \item[nonce:] A scalar value representing number of transactions associated with this account
  \item[gasPrice:] Default value is 0.02e12 Wei. Though clients can set their own gasPrice, most of the transactions use default value suggesting the high probability of its occurance.
  \item[gasLimit:] Maximum gas that can be used for executing this transaction
  \item[to:] Receiver's 160-bit address. For contract creation, a special symbol $\phi$ is used.
  \item[value:] Wei being transferred to the receipient.
  \item[v,r,s:] v =\{27,28\} and so can be efficiently encoded.
\end{description}

Additionally, contract creation has \textbf{init} field containing EVM-bytecode for initializing the contract.
We propose to compress the EVM bytecode using techniques presented in Section~\ref{sec:evmcompress}. 

Message call transaction has \textbf{data} field containing the inputs of the message call which includes methodid and arguments to the method being invoked.
Though the size is unlimited, existing transactions suggest that the input field has a high number of 0's~\cite{ethtx}. 
This is because ethereum uses 256-bit big integers but most of the transactions limit themselves to using 64-bit integers. 
This suggests some improvement possibilities.


%
\subsection{EVM ByteCode Compression}\label{sec:evmcompress}

Assigning optimal byte codes can reduce the storage for byte code~\cite{aslam2010}
The EVM can be modified to directly execute the compressed byte code~\citet{marc2003}.

Basic block compression techniques can yeild good compression rates~\cite{clausen2000}.
What are the common programming paradigms of smart contracts?

Write any novel contributions here.



