
\subsection{EVM ByteCode Compression}\label{sec:evmcompress}

There are several ways to compress EVM bytecode. Huffman encoding is probably the most straightforward
way to reduce bytecode size. By using Huffman encoding, we essentially compress instructions individually
by minimizing the number of bits taken by each opcode given how frequently they appear. One potential
benefit of using Huffman codes is that we can modify the EVM to directly execute the compressed bytecode,
without decoding the entire bytecode stream first. It, however, raises the challenge of making program counter modifications work
correctly, because now instructions are variable-length encoded.

Huffman codes reduce the size of individual instruction encoding, but do not take advantage of repetitive patterns in the bytecode.
A dictionary-based or grammar-based compression scheme can further reduce redundant information in the bytecode.
General-purpose compression algorithms like the LZ-family or Sequitur might just work,
but likely not optimal, because of the specific structure in bytecode sequences.
For example, some instructions take operands, sometimes very long operands.
Long operands are usually just random bits and are much harder to compress than the opcodes, which typically appear in patterns.
These long operands may confuse general-purpose compress algorithms, resulting in sub-optimal results.
We aim to devise a domain-specific compression scheme for EVM bytecode that can achieve better compression rates
by treating operands the opcodes separately.
