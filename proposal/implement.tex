Our plan is to implement the compression by modifying \emph{eth}:
 a C++ ethereum client. 
Eth serializes the data using a recursive length prefix (RLP) encoding.
A key/value database is used to store the transactions.

We plan to implement an encoder  that implements the compression  
discussed in Section~\ref{sec:blockcompress} and Section~\ref{sec:evmcompress} 
before serializing the data to database.
We now compress the data using general-purpose compressors like gzip and/or bzip2.

For decompression, we follow the reverse path and using
corresponding decoder in Eth, we  retrieve the original blockchain.

One question that we weren't able to decide early on is whether to apply general-compression techniques before or aftering storing the data to the database.
