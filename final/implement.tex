We implemented our compression scheme as an extension to the Go language implementation of the \eth{} client -- \texttt{go-ethereum}~\cite{go-eth}.
Our main compressor, which is written in Go, reads and parses the blockchain export file,
applies the reorganizations and field optimizations, and writes the blockchain in our custom compressed format to output.
We also have a python script that talks to the Go program to perform Huffman encoding for EVM bytecode.
We choose python due to limitations of Go when it comes to bit-level and non-byte-aligned operations.

To determine the proper Huffman codes to replace default EVM bytecode opcodes, we use a python program to scan the blockchain
and collect frequency information about the opcodes. The top 6 most frequently used opcodes and their corresponding encodings
are shown in~\autoref{tab:huffman-top6}.

\begin{table}[H]
\centering
\begin{tabular}{ c | c | c | c }
	Rank & {Opcode Name} & {Original Encoding} & {Huffman Encoding} \\
\hline
1 & push1   & 0110 0000 & 100\\
2 & stop    & 0000 0000 & 000\\
3 & add     & 0000 0001 & 0111\\
4 & swap1   & 1001 0000 & 0011\\
5 & suicide & 1111 1111 & 11111\\
6 & pop     & 0101 0000 & 11101\\
\end{tabular}
\caption{Top 6 most frequently used opcodes in EVM bytecode among all transactions.}
\label{tab:huffman-top6}
\end{table}

The Go compressor consists of around 350 lines of Go code,
with the python scripts adding an additional 270 lines of code.
Our source code is publically available online at
\url{https://github.com/huangyihe/eth-compressor}
and
\url{https://github.com/anithag/compressEVM}.

Our custom format performs the following optimizations compared to the original blockchain export format.

\paragraph{Field eliminations in block headers} We eliminate \textbf{parentHash}, \textbf{transactionsRoot},
and \textbf{difficulty} fields in block headers. These fields are redundant and can be computed from
other information stored in the block header.

\paragraph{Bloom filter compression} We compress the Bloom filter in \textbf{logsBloom} field using run-length
encoding to encode repetitive zero bytes.

\paragraph{Collapsing blocks by coinbase reward addresses} We group blocks by common coinbase reward addresses, and remove
the \textbf{coinbase} field in block headers. The coinbase is now a key for the ``block group'' within which all blocks
share the same coinbase reward address.

\paragraph{Collapsing transactions by recipient addresses} We group transactions within each block by recipient addresses, and
remove the \textbf{to} field in each transaction. Transactions are now organized in ``transaction groups'' where within
each group all transactions share the same recipient address.

\paragraph{Payload compression} EVM bytecode in payload is compressed using Huffman encoding. Message call data is compressed
using run-length encoding of zero bytes.

\paragraph{Extra data compression} We also discovered that the \textbf{extraData} field in a block header contains information
about the client version of the node that first announced the block to the network. Since there are far less client versions
than the growing number of blocks, we can compress it further by using a dictionary-like technique. We encode the
\textbf{extraData} fields using indexes to a table of all detected version strings.
