In this project we propose a transformation that rearranges the \eth{} blockchain resulting in about $40\%$ space savings on real blockchain.
We also demonstrate that existing compression programs benefit from the custom format. Additional gains of about $19.4\%$ in the best case and $10\%$ in the worst case are possible.
Our transformation adds a little overhead and 
the resulting runtime performance is comparable with that of gzip and bzip2 thout it outperforms in the case of xz. 
Below, we discuss a few possible future directions of the work.

Ethereum blockchain uses bloom filter to easily search the log entries of all transactions in a block.
Currently, we are only using a variant of run-length encoding to compress the Bloom filter, since it is sparse anyway. 
\citet{mitzenmacher:2001} described a more sophisticated Bloom filter compression technique by tuning the number of hash functions to be used. 
In future, we plan to use this technique to further reduce the number of bits to be broadcast to full nodes.
In particular, we think that coming up with a different number of hash functions (go-ethereum uses 3 hash functions) can compress the Bloom filter more effectively.

We also plan to focus on improving the runtime performance of our tool for compressing and decompressing the blockchain.
Using techniques for hiding the I/O latencies of file I/O, we can amortize 
the overhead incurred for decompressing the blockchain.


