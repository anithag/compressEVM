
\eth{} is a public blockchain-based cryptocurrency and supports Turing-complete smart contracts~\cite{ethereum}.
The blockchain maintains a list of records, called \emph{blocks} and serves as a ledger for all transactions.
The value token of the blockchain is called \emph{ether} and 
is traded in cryptocurrency exchanges.
Clients use ether to pay for executing and maintaining the transactions.~\footnote{
\emph{Finney}, \emph{Szabo}, and \emph{Wei} are the denominations of ether.}
There are two kinds of \eth{} accounts: 
\renewcommand\labelenumi{(\theenumi)}
\begin{enumerate*}
	\item \emph{Externally owned accounts} (EOA) and
	\item \emph{Contract accounts}.
\end{enumerate*}
Contract accounts are special accounts containing the smart contracts in bytecode format, referred to as \emph{EVM bytecode}.
The bytecode is stack-based and is executed on the \emph{Ethereum Virtual Machine} (EVM).
Contract accounts comprise around $25\%$ of all \eth{} accounts.~\footnote{
At the time of writing, number of EOA and contract accounts are 555,043 and 197,261 respectively.}  

%Currently there are around 2,488,503 blocks with size ranging between 1.5KB to 12.5KB~\cite{ethblocksize}.  
%The blockchain is growing at a rate of 1.2GB/month and
%the space required for storing full blockchain is as high as 60GB~\cite{ethdiskspace}.
%Compressing the blockchain can therefore be useful for full client nodes (e.g., IoT, embedded systems with less storage)
%that download the entire blockchain.

An \eth{} node needs a copy of the entire blockchain to fully participate in the network.
As the blockchain grows, the time it takes to perform blockchain synchronization from scratch also grows.
A simple Google search about \eth{} blockchain synchronization issues yields many results of users complaining about slow syncing time.
One proposed solution is to offer the full blockchain for download as a file.
The client can then use the file to perform database update offline.
The file size, which is around 3.3 GB at the time of writing, may also soon become cumbersome for Internet transfer.
Compressing the blockchain in this format can therefore be useful to distribute the blockchain more efficiently,
especially for IoT/embedded devices with limited network capabilities.

In this project, we propose to compress the \eth{} blockchain by leveraging domain-specific information.
One could use the general-purpose compression software (e.g., gzip, bzip2, 7zip) and compress the blockchain.
However, a majority of blockchain contains 256-bit hashes which, if we assume are perfectly random, will be hard to compress.
Also these programs cannot exploit certain properties that are specific to blockchain and thus do not offer optimal outputs.
Our goal is to come-up with compression techniques that  are specialized for blockchain.
While we aim to provide a comparative study, we do not conjecture that our compression algorithm always outperforms the
existing mature compressors.
We believe that the insights can be added as an extension to be able to get effective compression rates.

%Blockchain compression involves compressing each block.
%Although compressing a block (and hence transactions) also includes compressing EVM bytecode, 
%for the sake of exposition we view them as complementary tasks. 
%In particular, techniques to compress bytecode are orthogonal to techniques for compressing other fields of transaction.
%In Section~\ref{sec:blockcompress} and 
%Section~\ref{sec:evmcompress}, we   
%outline the opportunities for compressing \eth{} block and EVM bytecode.
