
\eth{} is a public blockchain based cryptocurrency and supports Turing-complete smart contracts~\cite{ethereum}.
The blockchain maintains a list of records, called \emph{blocks} and serves as a ledger for all transactions.
The value token of the blockchain is called \emph{ether} and 
is traded in cryptocurrency exchanges.
Clients use ether to pay for executing and maintaining the transactions.~\footnote{
\emph{Finney}, \emph{Szabo}, and \emph{Wei} are the denominations of ether.}
There are two kinds of \eth{} accounts: 
\renewcommand\labelenumi{(\theenumi)}
\begin{enumerate*}
	\item \emph{Externally owned accounts} (EOA) and
	\item \emph{Contract accounts}.
\end{enumerate*}
Contract accounts are special accounts containing the smart contracts in bytecode format, referred to as \emph{EVM bytecode}.
The bytecode is stack-based and is executed on \emph{Ethereum Virtual Machine} (EVM).
Contract accounts comprise around $25\%$ of the entire \eth{} accounts.~\footnote{
At the time of writing, number of EOA and contract accounts are 555,043 and 197,261 respectively.}  
 
Currently there are around 2,488,503 blocks with size ranging between 1.5KB to 12.5KB~\cite{ethblocksize}.  
The blockchain is growing at a rate of 1.2GB/month and
the space required for storing full blockchain is as high as 60GB~\cite{ethdiskspace}.
Compressing the blockchain can therefore be useful for full client nodes (e.g., IoT, embedded systems with less storage)
that download the entire blockchain.

In this project, we propose to compress the entire \eth{} blockchain by leveraging the domain-specific information.
One could use the general-purpose compression software (e.g., gzip, bzip2, 7zip) and compress the blockchain.
However, a majority of blockchain contains 256-bit hashes which, if we assume are perfectly random, will be hard to compress.
Also these programs cannot exploit certain properties that are specific to blockchain and thus do not offer optimal outputs.
Our goal is to come-up with compression techniques that  are specialized for blockchain.
While we aim to provide a comparative study, we do not conjecture that our compression algorithm always outperforms the existing mature compressors.
We believe that the insights can be added as an extension to be able to get effective compression rates.   

Blockchain compression involves compressing each block.
Although compressing a block (and hence transactions) also includes compressing EVM bytecode, 
for the sake of exposition we view them as complementary tasks. 
In particular, techniques to compress bytecode are orthogonal to techniques for compressing other fields of transaction.
In Section~\ref{sec:blockcompress} and 
Section~\ref{sec:evmcompress}, we   
outline the opportunities for compressing \eth{} block and EVM bytecode. 

\subsection{Ethereum Block Compression}\label{sec:blockcompress}

A block is comprised of a block header, ommer (parent) block headers and a series of transactions. 
A block header stores cryptographic hashes of all transactions in the block as well as of previous and parent blocks.
These hashes play an important role for verifying the validity of blocks.
It might seem that there is not enough information to compress these random bits. 
The block hashes, however, have a fixed number of leading zeros and
can therefore be encoded more efficiently.
Also, a closer inspection at the block header reveals that 
some information is either redundant or can be computed given other fields of the block.
We explain how to exploit this redundancy to efficiently encode the block information.

A block header has the following format. We outline the possibility of efficiently encoding some fields. 
\begin{description}
 \item[parentHash] 
 \item[ommersHash:]  Keccak 256-bit hash.
 \item[stateRoot]  
 \item[transactionsRoot]
 \item[beneficiary:]160-bit address.
 \item[logsBloom:] Bloom filter for log entries.
 \item[difficulty:] Can be computed from previous block using the formula
 $$
 D_{c} = D_{p} + \dfrac{D_{p}}{2048\times\textrm{max}\left(\dfrac{1 - (ts_{c} - ts_{p})}{10}, -99\right)} + 2^{b/10^5-2}
 $$
 where $ts$ denotes timestamps, subscripts $c$ and $p$ denote current and parent blocks respectively, and $b$ is the block number.
 This suggests the possibility of saving bits without having to store difficulty for all blocks.
 \item[number:] Number of ancestor blocks
 \item[gasLimit:] Scalar value limiting the gas expenditure
 \item[gasUsed:] Scalar value indicating the total gas used by all transactions in the block. This information can encoded efficiently.
 \item[timestamp]
 \item[extraData]
 \item[mixHash:] 256-bit hash
 \item[nonce:] 64-bit hash
\end{description}

An efficient encoder can exploit that \textbf{difficulty} and \textbf{gasUsed} can computed on the fly and therefore save around 512 bits of information per block (Recall that \eth{} uses 256-bit integers by default).  


A transaction is a signed message. 
Transactions are much more compressible and can be a major source of compression gains.
There are two types of transactions: {\em Contract Creation} transactions and {\em Message Call} transactions.
Both types have the following common fields:
\begin{description}
  \item[nonce:] A scalar value representing number of transactions associated with this account
  \item[gasPrice:] Default value is $0.02\times10^{12}$ Wei. Though clients can set their own gasPrice, most of the transactions use default value suggesting the high probability of its occurance.
  \item[gasLimit:] Maximum gas that can be used for executing this transaction
  \item[to:] Receiver's 160-bit address. For contract creation, a special symbol $\phi$ is used.
  \item[value:] Wei being transferred to the recipient.
  \item[v,r,s:] v =\{27, 28\} and so can be efficiently encoded.
\end{description}

Additionally, a Contract Creation transaction has an \textbf{init} field containing the EVM bytecode for initializing the contract.
We propose to compress the EVM bytecode using techniques presented in Section~\ref{sec:evmcompress}.

A Message Call transaction has \textbf{data} field that contains the inputs of the message call: a method ID and parameters supplied to the invoked method.
Existing transactions suggest that most of the inputs are going to a high number of 0's~\cite{ethtx}. 
This is because \eth{} uses 256-bit big integers, but most of the transactions limit themselves to using 64-bit integers.
This suggests that small numbers can be encoded efficiently.


%
\subsection{EVM ByteCode Compression}\label{sec:evmcompress}

There are several ways to compress EVM bytecode. Huffman encoding is probably the most straightforward
way to reduce bytecode size. By using Huffman encoding, we essentially compress instructions individually
by minimizing the number of bits taken by each opcode given how frequently they appear. One potential
benefit of using Huffman codes is that we can modify the EVM to directly execute the compressed bytecode,
without decoding the entire bytecode stream first. It, however, raises the challenge of making program counter modifications work
correctly, because now instructions are variable-length encoded.

Huffman codes reduce the size of individual instruction encoding, but do not take advantage of repetitive patterns in the bytecode.
A dictionary-based or grammar-based compression scheme can further reduce redundant information in the bytecode.
General-purpose compression algorithms like the LZ-family or Sequitur might just work,
but likely not optimal, because of the specific structure in bytecode sequences.
For example, some instructions take operands, sometimes very long operands.
Long operands are usually just random bits and are much harder to compress than the opcodes, which typically appear in patterns.
These long operands may confuse general-purpose compress algorithms, resulting in sub-optimal results.
We aim to devise a domain-specific compression scheme for EVM bytecode that can achieve better compression rates
by treating operands the opcodes separately.


