\documentclass[12pt]{article}
\usepackage{amsmath, amsfonts}
\usepackage{graphicx}
\usepackage{hyperref}
\usepackage{array}
\usepackage{multirow}
\usepackage{stmaryrd}
\usepackage{mathpartir}
\usepackage{amsthm}
\usepackage{xspace}
\usepackage{tabularx}
\usepackage{amssymb}
\usepackage{amstext}
\usepackage{color, float}
\usepackage{wrapfig}
\usepackage[inline]{enumitem}
\usepackage{listings}


\newcommand\peth{Ethereum}

\title{Ethereum Blockchain Compression\\
	\large CS222: Project Update
	}

\author{Anitha Gollamudi 
	\and
	Yihe Huang}
\begin{document}
%\input{defs}

\maketitle

\section{Summary}
In this project, we propose to compress the entire \peth blockchain by leveraging the domain-specific information.
One could use the general-purpose compression software (e.g., gzip, bzip2, 7zip) and compress the blockchain.
However, a majority of blockchain contains 256-bit hashes which, if we assume are perfectly random, will be hard to compress.
Also these programs cannot exploit certain properties that are specific to blockchain and thus do not offer optimal outputs.
Our goal is to come-up with compression techniques that  are specialized for blockchain.

We emphasize that outperforming the existing mature compressors is not our primarily goal, rather
we hope that the insights from this project can be added as an extension to be able to get effective compression rates.   

\section{Progress}
Our project execution plan involves two phases: 
\begin{enumerate*}
	\item Compress the EVM byte code in contract creation transactions 
	\item Compress the fields of transactions leveraging the block structural information
\end{enumerate*}
So far, we focused our efforts on getting the EVM byte code compression which we describe in detail.

After setting up our Ethereum clients, we downloaded the Morden testnet blockchain which occupies around 25GB of disk-space.\footnote{We use testnet as our first approximation of the actual \peth blockchain.}
After vetting through the format of transactions, we identified that out of the 4.5million transactions,  104277 ($\sim$2\%) create contracts.
The ratio is roughly representative of the contract creation transactions in the actual blockchain.

Extracting byte code from contracts is fairly straight-forward. Our sub-goal is to compress this byte code.
One of our observations is that the encoding of EVM opcodes is not information-theoretically optimal. All opcodes consume a byte.
So our first natural attempt is to obtain optimal encoding such that frequently used opcodes can be encoded using fewer bits.

\begin{table}[H]
\centering
\begin{tabular}{c | c | p{4cm} | p{4cm} }
	opcode & frequency & {Original Encoding} & {Huffman Encoding}  \\
\hline
push1 & 30668777 & 0110 0000 & 100\\
stop & 23667846 & 0000 0000 & 000 \\
add  & 15255847 & 0000 0001  & 0111\\
swap1  & 12899156 &1001 0000 & 0011 \\
suicide & 12279587 &1111 1111 & 11111 \\
pop    & 10986684 &0101 0000 & 11101 \\
\hline
\end{tabular}
\caption{Top 6 opcodes}
\label{tab:freqopcodes}
\end{table}

We used huffman encoding to obtain optimal encodings.
Table~\ref{tab:freqopcodes} shows the frequencies of the top 6-most used opcodes, corresponding original encodings and Huffman codes.
With the new encoding the byte code could be compressed by about $31\%$. 
Note that the below numbers represent the size of byte code, not that of blockchain.

\begin{lstlisting}
Original Size (bytes): 271535767
Compressed Size : 187170859
Compression ratio : 0.689
\end{lstlisting}

This seems encouraging but a comparison with gzip shows that there is more room for compression. We describe the list of optimizations we plan to
do in Section~\ref{sec:next}.
\begin{lstlisting}
Compressed Size (gzip) : 45965241
Compression ratio : 0.169
\end{lstlisting}

We also extracted transactions that invoke the contracts. 
These transactions have a field \textbf{input} which is used to send input parameters to the method being invoked.
Much to our surprise, the total size of the data inside the \textbf{input} is larger (2x) 
than the total size of the byte code. Most of the data is redundant because EVM uses 256-bit integers, but in practice only small integers are used.
This is easily amenable to compression.

\section{Next Steps}\label{sec:next}

While the results are encouraging, we can improve over Huffman codes by using more sophiscated compression mechanisms. As mentioned
in our proposal, we will try to apply LZ-family compression algorithms to identify repetitive patterns in the program. To achieve
better results than generic compression algorithms such as gzip, we will separate opcodes and long operands to avoid polluting
compression algorithms with random bits generated by cryptographic hash functions.

We also plan to implement the phase-2 of our execution plan: to compress the fields of the blocks and transactions as described in our
proposal. We also realize that the blockchain data obtained from the testnet may not be representative of that on the real Ethereum network
(the main chain). The main chain is much larger in size and uses more sophisticated contracts, and we are currently downloading it and doing the same analyses on the
Ethereum main chain as well.

\end{document}
